\usepackage[
nochapters, % Turn off chapters since this is an article        
beramono, % Use the Bera Mono font for monospaced text (\texttt)
eulermath,% Use the Euler font for mathematics
pdfspacing, % Makes use of pdftex’ letter spacing capabilities via the microtype package
dottedtoc % Dotted lines leading to the page numbers in the table of contents
]{classicthesis} % The layout is based on the Classic Thesis style

\usepackage{arsclassica} % Modifies the Classic Thesis package

\usepackage[T1]{fontenc} % Use 8-bit encoding that has 256 glyphs

\usepackage[utf8]{inputenc} % Required for including letters with accents

\usepackage{graphicx} % Required for including images
\graphicspath{{Figures/}} % Set the default folder for images

\usepackage{enumitem} % Required for manipulating the whitespace between and within lists

\usepackage{lipsum} % Used for inserting dummy 'Lorem ipsum' text into the template

\usepackage{subfig} % Required for creating figures with multiple parts (subfigures)

\usepackage{amsmath,amssymb,amsthm} % For including math equations, theorems, symbols, etc
\usepackage{placeins}
\usepackage{graphicx}
\usepackage[italian]{babel}
\usepackage{varioref}

\theoremstyle{plain}
\newtheorem{teorema}{teorema}
\theoremstyle{plain}
\newtheorem{esercizio}{esercizio}
\theoremstyle{plain}
\newtheorem{applicazione}{applicazione}
\theoremstyle{plain}
\newtheorem{osservazione}{osservazione}
\theoremstyle{plain}
\newtheorem{assioma}{assioma}
\theoremstyle{plain}
\newtheorem{postulato}{postulato}
\theoremstyle{plain}
\newtheorem{corollario}{corollario}[teorema]
\theoremstyle{plain}
\newtheorem{lemma}[teorema]{lemma}
\theoremstyle{definition}
\newtheorem{definizione}{definizione}
\theoremstyle{definition}
\newtheorem{dimostrazione}{dimostrazione}
\theoremstyle{plain}
\newtheorem{esempio}{esempio}
\theoremstyle{plain}
\newtheorem{proposizione}{proposizione}
\makeatletter
\def\d{\ensuremath\partial}
\def\d{\ensuremath\partial\ }
\def\n{\ensuremath\nabla}
\def\n{\ensuremath\nabla\ }
\def\th@plain{%
	\thm@notefont{}% same as heading font
	\itshape % body font
}
\def\th@definizione{%
	\thm@notefont{}% same as heading font
	\normalfont % body font
}
\hypersetup{
colorlinks=true, breaklinks=true, bookmarks=true,bookmarksnumbered,
urlcolor=webbrown, linkcolor=RoyalBlue, citecolor=webgreen, % Link colors
pdftitle={}, % PDF title
pdfauthor={\textcopyright}, % PDF Author
pdfsubject={}, % PDF Subject
pdfkeywords={}, % PDF Keywords
pdfcreator={pdfLaTeX}, % PDF Creator
pdfproducer={LaTeX with hyperref and ClassicThesis} % PDF producer
}